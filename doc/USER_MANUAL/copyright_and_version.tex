%%%%%%%%%%%%%%%%%%%%%%%%%%%%%%%%%%%%%%%%%%%%%%%%%

\chapter*{Copyright}
\addcontentsline{toc}{chapter}{Copyright}

Main historical authors: Dimitri Komatitsch and Jeroen Tromp

Princeton University, USA, and CNRS / University of Marseille, France\\
$\copyright$ Princeton University and CNRS / University of Marseille, July 2012\\

\noindent
This program is free software; you can redistribute it and/or modify
it under the terms of the GNU General Public License as published
by the Free Software Foundation (see Appendix \ref{cha:License}).\\

\noindent
Please note that by contributing to this code, the developer understands and agrees that this project and contribution
are public and fall under the open source license mentioned above.\\

\noindent
\textbf{\underline{Evolution of the code:}}\\

version 7.0, Dimitri Komatitsch, Zhinan Xie, Paul Cristini, Roland Martin and Rene Matzen, July 2012:\\
added support for Convolution PML absorbing layers;
added higher-order time schemes (4th order Runge-Kutta and LDDRK4-6);
many small or moderate bug fixes\\

version 6.2, many developers, April 2011:\\
restructured package source code into separate src/ directories;
added configure \& Makefile scripts and a PDF manual in doc/;
added user examples in EXAMPLES/;
added a USER\_T0 parameter to fix the onset time in simulation\\

version 6.1, Christina Morency and Pieyre Le Loher, March 2010:\\
added SH (membrane) waves calculation for elastic media;
added support for external fully anisotropic media;
fixed some bugs in acoustic kernels\\

version 6.0, Christina Morency and Yang Luo, August 2009:\\
support for poroelastic media;
adjoint method for acoustic/elastic/poroelastic\\

version 5.2, Dimitri Komatitsch, Nicolas Le Goff and Roland Martin, February 2008:\\
support for CUBIT and GiD meshes;
MPI implementation of the code based on domain decomposition with METIS or SCOTCH;
general fluid/solid implementation with any number, shape and orientation of matching edges;
fluid potential of density $*$ displacement instead of displacement;
absorbing edges with any normal vector;
general numbering of absorbing and acoustic free surface edges;
cleaned implementation of attenuation as in Carcione (1993);
merged loops in the solver for efficiency;
simplified input of external model;
added CPU time information;
translated many comments from French to English\\

version 5.1, Dimitri Komatitsch, January 2005:\\
more general mesher with any number of curved layers;
Dirac and Gaussian time sources and corresponding convolution routine;
option for acoustic medium instead of elastic;
receivers at any location, not only grid points;
moment-tensor source at any location, not only a grid point;
color snapshots;
more flexible DATA/Par\_file with any number of comment lines;
Xsu scripts for seismograms;
subtract t0 from seismograms;
seismograms and snapshots in pressure in addition to vector field\\

version 5.0, Dimitri Komatitsch, May 2004:\\
got rid of useless routines, suppressed commons etc.;
weak formulation based explicitly on stress tensor;
implementation of full anisotropy;
implementation of attenuation based on memory variables\\

based on SPECFEM2D version 4.2, June 1998\\
(c) by Dimitri Komatitsch, Harvard University, USA
and Jean-Pierre Vilotte, Institut de Physique du Globe de Paris, France

itself based on SPECFEM2D version 1.0, 1995\\
(c) by Dimitri Komatitsch and Jean-Pierre Vilotte,
Institut de Physique du Globe de Paris, France


