%------------------------------------------------------------------------------------------------%

\chapter{Troubleshooting}\label{cha:Troubleshooting}

%------------------------------------------------------------------------------------------------%


\section*{FAQ}

\subsection*{Regarding the structure of some of the database files}

\paragraph{Question:} Can anyone tell me what the columns of the SPECFEM2D boundary
condition files in \texttt{SPECFEM2D/DATA/}~\\
\texttt{Mesh\_canyon} are?
%
\begin{verbatim}
    SPECFEM2D/DATA/Mesh_canyon/canyon_absorbing_surface_file
    SPECFEM2D/DATA/Mesh_canyon/canyon_free_surface_file
\end{verbatim}

\paragraph{Answer:} \texttt{canyon\_absorbing\_surface\_file} refers to parameters related to the
absorbing conditions:
The first number (180) is the number of absorbing elements (nelemabs in the
code).
Then the columns are:
\begin{enumerate}[wide=1em, label=column \arabic*:]
\item the element number
\item the number of nodes of this element that form the absorbing surface
\item the first node
\item the second node
\end{enumerate}

\texttt{canyon\_free\_surface\_file} refers to the elements of the free surface
(relevant for enforcing free surface condition for acoustic media):
The first number (160) is the number of  elements of the free surface.
Then the columns are (similar to the absorbing case):
\begin{enumerate}[wide=1em, label=column \arabic*:]
\item the element number
\item the number of nodes of this element that form the absorbing surface
\item the first node
\item the second node
\end{enumerate}

Concerning the free surface description file, nodes/edges pertaining to
elastic elements are discarded when the file is read (if for whatever
reason it was simpler to include all the nodes/edges on one side of a
studied area and that there are among them some elements that are
elastic elements, only the nodes/edges of acoustic elements are kept).

These files are opened and read in \texttt{meshfem2D.F90} using subroutines
\texttt{read\_abs\_surface()} and \texttt{read\_}~\\
\texttt{acoustic\_surface()}, which are in \texttt{part\_unstruct.F90}


